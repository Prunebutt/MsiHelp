% -----------------------------------------------------------------
%	 Non parametric and frequency domain identification methods
% -----------------------------------------------------------------
\begin{tcolorbox}[colback=brown!5!white,colframe=brown!75!black,title=\textbf{Fourier Transformation}]
	\textbf{FT:}\begin{align*}
				  &F\{F\}(\omega) = \int_{-\infty}^{\infty} f(t) e^{-j\omega t}\diff t\\
				  &G(j\omega)=\frac{Y_0}{U_0}e^{j\omega t}
	\end{align*}
	\textbf{iFT:}\begin{align*}
	f(t) = F^{-1}\{F\}(t) = \frac{1}{2\pi}\int_{-\infty}^{\infty} F(\omega) e^{-j\omega t}d\omega
	\end{align*}
	\textbf{DFT:}\begin{align*}
	U(m) := \sum_{k = 0}^{N-1} u(t)e^{-j\frac{2\pi m k}{N}}
	\end{align*}
	\textbf{iDFT:}\begin{align*}
	u(n) := \sum_{k = 0}^{N-1} U(k)e^{j\frac{2\pi k n}{N}}
				  \end{align*}
\end{tcolorbox}

\begin{tcolorbox}[colback=brown!5!white,colframe=brown!75!black,title=\textbf{Useful frequency things}]
  $\omega = 2\pi f = \frac{2\pi}{T}$, \quad $f_s > 2 \ti{f}{max}$, \quad $T = N \Delta{t} = \frac{N}{f_s} = (\ti{f}{res})^{-1} $\quad\\
  $\sin(\varphi)=\frac{e^{j\varphi}-e^{-j\varphi}}{2},\ \cos(\varphi)=\frac{e^{j\varphi}+e^{-j\varphi}}{2}$
\end{tcolorbox}

\begin{tcolorbox}[colback=brown!5!white,colframe=brown!75!black,title=\textbf{Aliasing and Leakage Errors}]
\begin{description}
	\item[Aliasing Error:] Due to sampling of continous signal to discrete signal. Avoid with Nyquist Theoreme:
	\begin{equation*}
	\ti{f}{Nyquist} = \frac{1}{2\Delta t} [\SI{}{\hertz}] \quad or \quad \ti{\omega}{Nyquist} = \frac{2\pi}{2\Delta t} [\SI{}{\radian \per \second}]
	\end{equation*}
	
	\item[Leakage Error:] Due to windowing.
	\begin{equation*}
	\ti{\omega}{base} := \frac{2\pi}{N \cdot \Delta t} = \frac{2\pi}{T} \rightarrow \omega = m\cdot \frac{2\pi }{N \cdot \Delta t}
	\end{equation*}
\end{description}
\end{tcolorbox}

\begin{tcolorbox}[colback=brown!5!white,colframe=brown!75!black,title=\textbf{Crest Factor (ger. = Scheitelfaktor)}]
\begin{flalign*}
  \text{Crest Factor } =\frac{\ti{u}{max}}{\ti{u}{rms}} \quad \text{with}: \ti{u}{rms} &:= \sqrt{\frac{1}{T} \int_{0}^{T} u(t)^2 \, \diff t} \underbrace{\left(= \sqrt{\frac{u_1^2+u_2^2}{2}}\right)}_{\mathrm{symm.\ square\ wave}}%
  &\\
	\quad \text{and} \quad \ti{u}{max} &:=\max_{t \in [0,T]} |u(t)| &
\end{flalign*}
\end{tcolorbox}

\begin{tcolorbox}[colback=brown!5!white,colframe=brown!75!black,title=\textbf{Optimising Multisine for optimal crest factor}]
\begin{description}
	\item[Frequency:] Choose frequencies in logarithmic manner as multiples of the base frequency. $\quad \omega_{k+1}/\omega_k \approx 1.05/1.01/1.03$ (round to $n\cdot\ti{\omega}{base}$!)

	\item[Phase:] To prevent high peaks (Crest Factor): random algorithm for phase shifts
\end{description}
\end{tcolorbox}

\begin{tcolorbox}[colback=brown!5!white,colframe=brown!75!black,title=\textbf{Multisine Identification Implementation procedure}]
\begin{description}
	\item[Window Length:] Integer multiple of sampling time: $T = N \cdot \Delta t$

	\item[Harmonics of base frequency:] Are contained in multisine \\ $\omega_{base} = \frac{2\pi}{T}$

	\item[Highest contained Frequency:] Is \textbf{half} of Nyquist frequency: $\ti{\omega}{Nyquist} = \frac{2\pi}{4\Delta T}$

	\item[Experiment and Analysis:] (Step 2):
	  \begin{enumerate}
	  \item Insert Multisine periodically
	  \item Drop first Periods (till transients died out) 
	  \item Record $M$ Periods, with $N$ samples, of input and output data 
	  \item Average all windows and apply DFT (or vice versa) 
	  \item Build transfer function: $ \hat{G}_{{j\omega}_{k}} = \frac{\hat Y(k(p))}{\hat U (k(p))}$
	  \end{enumerate}
	  
\end{description}
\end{tcolorbox}

\begin{tcolorbox}[colback=brown!5!white,colframe=brown!75!black,title=\textbf{Nonparametric and Frequency Domain Identification Models}]
\textbf{Impulse response and transfer function:}
\begin{align*}
	y(t) &= \int_{0}^{\infty} g(\tau) u(t-\tau) \, \diff \tau \\
	Y(s) &= G(s)\cdot U(s) \\
	G(s) &= \int_{0}^{\infty} e^{-st} g(t) \, \diff t 
\end{align*}
\textbf{Bode diagram from frequency sweeps:}
\begin{align*}
	u(t) &= A \cdot \sin(\omega \cdot t),\quad y(t) = \lVert G (j\cdot \omega )\rVert A \cdot \sin(\omega \cdot t + \alpha)
\end{align*}
\end{tcolorbox}

\begin{tcolorbox}[colback=brown!5!white,colframe=brown!75!black,title=\textbf{Bode Diagramm}]
	Magnitude $\widehat{=}$ Amplitude $|G(j\omega)|$\\
	Phase $\widehat{=}$ $\arg \, G(j\omega)$
	\todo[inline]{Hier sollten wir glaub noch bisschen was rein machen.}
  \end{tcolorbox}

  \begin{tcolorbox}[colback=brown!5!white,colframe=brown!75!black,title=\textbf{When to use what in frequency response}]
	\begin{itemize}
	\item In general: Multisines are a good approach
	\item Very fast system and transients $\Rightarrow$ Frequency sweep
	\item \textbf{Very} slow system $\Rightarrow$ Step response
	\item In the middle $\Rightarrow$ Multisines
	\end{itemize}
  \end{tcolorbox}


% \begin{tcolorbox}[colback=violet!5!white,colframe=violet!75!black,title=\textbf{Recursive Least Squares}]
% \begin{flalign*}
% 	& \textbf{New Inverse Covariance: } Q_K = Q_{k-1} + \phi_K \phi_K^T &
% \end{flalign*}
% \end{tcolorbox}

\begin{tcolorbox}[colback=violet!5!white,colframe=violet!75!black,title=\textbf{Kalman Filter}]
\textbf{Valid for Discrete and Linear!}
\begin{flalign*}
	& \text{If recursive least squares: } x_{ k+1 }\, =\, A_{ k }\cdot x_{ k }& \\
	& x_{k+1} = A_k \cdot x_k + \omega_k \quad \text{and} \quad y_k = C_k \cdot x_k + v_k&
\end{flalign*}

\textbf{Steps of Kalman Filter}
\begin{description}
	\item[\small 1 Prediction] 
		\begin{flalign*}
		& \hat x_{[k | k-1]} = A_{k-1} \cdot \hat x_{[k-1 | k-1]} & \\ 
		& P_{[k | k-1]} = A_{k-1} \cdot P_{[k-1 | k-1]} \cdot A_{k-1}^{T} \cdot W_{k-1} & \\
		& \text{If RLS} \Rightarrow \mathrm{no\ } W_{k-1} &
		\end{flalign*}
		
	\item[\small 2 Innovation update] 
		\begin{flalign*}
		& P_{[k | k]} = (P_{[k | k-1]}^{-1} + C_{k}^{T} \cdot V^{-1} \cdot C_k)^{-1} & \\ 
		& \hat x_{[k | k]} = \hat x_{[k | k-1]} + P_{[k | k]} \cdot C_{k}^{T} \cdot V^{-1} \cdot (y_k - C_k \cdot \hat x_{[k | k-1]}) &
		\end{flalign*}
\end{description}
\end{tcolorbox}

\begin{tcolorbox}[colback=violet!5!white,colframe=violet!75!black,title=\textbf{Useful hints and practices}]
  \begin{itemize}
  \item Time invariant: $t$ is only an argument of $u(*)$ or $y(*)$ i.e. \textbf{not}:
	\begin{equation*}
	  \dot{y}(t)=u(t)+\cos(t),\ \dot{y}(t)=t\cdot u(t) + y(t)
	\end{equation*}
  \item Linear: highest exponent of $u$ and $y$ is 1, i.e. \textbf{not}:
	\begin{equation*}
	  \dot{y}(t)=u(t)^2
	\end{equation*}
  \item Affine: linear and has additive term \textbf{independent} from $u$ and $y$, i.e.:
	\begin{equation*}
	  \dot{y}(t)=y(t)+\cos(t),\ \dot{y}(t)=y(t)+C
	\end{equation*}
  \item Sample Mean: $\hat{\theta}_N(Y_N)=\frac{1}{N}\sum^N_{k=1}Y(k)$
  \item Sample Variance: $S^2=\frac{1}{N-1}\sum^N_{k=1}(Y(k)-\mathbb{E}\{Y_N\})^2$

  \end{itemize}
\end{tcolorbox}

%%% Local Variables:
%%% mode: latex
%%% TeX-master: "../HelpSheet"
%%% End:
