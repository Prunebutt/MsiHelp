ew% -----------------------------------------------------------------
%	 Non parametric and frequency domain identification methods
% -----------------------------------------------------------------
\begin{tcolorbox}[colback=brown!5!white,colframe=brown!75!black,title=\textbf{Fourier Transformation}]
\textbf{How to compute FT?} By DFT, which solves the problem of finite time and discrete values.

\textbf{Can we use an input with many frequencies to get many FRF (Frequency Response Function) values in a single experiment?} So far only frequency sweeping (high comp. times due to repetition for each frequency). We should use multisines!
\end{tcolorbox}

\begin{tcolorbox}[colback=brown!5!white,colframe=brown!75!black,title=\textbf{Aliasing and Leakage Errors}]
\begin{description}
\item[Aliasing Error] due to sampling of continous signal to discrete signal. Avoid with Nyquist Theoreme:

\begin{equation*}
f_{Nyquist} = \frac{1}{2\Delta t} [Hz] \quad or \quad \omega_{Nyquist} = \frac{2 \pi }{ 2\Delta t} [rad/s]
\end{equation*}

\item[Leackage Error] due to windowing.
\begin{equation*}
\omega _{ base }:=\frac { 2\pi  }{ N\cdot \Delta t } =\frac { 2\pi  }{ T }  \rightarrow \omega = m\frac { 2\pi  }{ N\cdot \Delta t }
\end{equation*}
\end{description}
\end{tcolorbox}

\begin{tcolorbox}[colback=brown!5!white,colframe=brown!75!black,title=\textbf{Crest Factor = Scheitelfaktor}]
\begin{equation*}
CrestFactor=\frac { u_{ max } }{ u_{ rms } } \quad with:
\end{equation*}
\begin{equation*}
u_{ rms }\, := \sqrt{ \frac { 1 }{ T } \int _{ 0 }^{ T }{ u(t)^2 \, dt }  } \quad and \quad u_{ max }\, :=\underset { t\, \in \, [0,T] }{ max } |u(t)|
\end{equation*}
\end{tcolorbox}

\begin{tcolorbox}[colback=brown!5!white,colframe=brown!75!black,title=\textbf{Optimising Multisine for optimal crest factor}]
\begin{description}
\item[Frequency:] Choose frequencies in logarithmic manner as multiples of the base frequency. $\qquad \omega_{k+1}/\omega_{k} \approx 1.05$
\item[Phase:] To prevent high peaks (Crest Factor) in the Signal, the phases of the different frequencies are modulated accordingly. (Positive interference)
\end{description}
\end{tcolorbox}

\begin{tcolorbox}[colback=brown!5!white,colframe=brown!75!black,title=\textbf{Multisine Identification Implementation procedure}]
\begin{description}
\item[Window Length] integer multiple of sampling time:  \( T = N \cdot \delta t\)
\item[Harmonics of base frequency] are contained in multisine \\ \( {\omega}_{base} = \frac{2 \pi}{T}\)
\item[Highest contained Frequency] is \textbf{half} of Nyquist frequency: \( {\omega}_{Nyquist}  = \frac{2 \pi}{4 \Delta T}\)

\item[Experiment and Analysis] (step 2): Insert Multisine periodically. Drop first Periods (till transients died out). Record M Periods, each with N samples, of input and output data. Average all the M periods and make the DFT (or vice versa). Finally build transfer function.
\end{description}

\begin{equation*}
{\hat{G} _{j{\omega}_{k}}}=\frac{\hat{Y}(k(p))}{\hat{U}(k(p))}
\end{equation*}
\end{tcolorbox}




\begin{tcolorbox}[colback=brown!5!white,colframe=brown!75!black,title=\textbf{Nonparametric and Frequency Domain Identification Models}]
Impulse response and transfer function

\begin{equation*}
y(t)=\int _{ 0 }^{ \infty  }{ g(\tau)u(t-\tau) \delta t } 
\end{equation*}

\begin{equation*}
Y(s)=G(s)\cdot U(s)
\end{equation*}

\begin{equation*}
G(s)=\int _{ 0 }^{ \infty  }{ \epsilon }^{ -st }g(t)dt
\end{equation*}

Bode diagram from frequency sweeps
\begin{equation*}
u(t)=A\cdot sin(\omega \cdot t),\quad y(t)=\parallel G(j\cdot \omega )\parallel A\cdot sin(\omega \cdot t+\alpha )
\end{equation*}
\end{tcolorbox}


\begin{tcolorbox}[colback=brown!5!white,colframe=brown!75!black,title=\textbf{Bode Diagramm}]
Magnitude = Amplitude $|G(j\omega)|$\\
Phase $arg \, G(j\omega)$
\end{tcolorbox}


\begin{tcolorbox}[colback=violet!5!white,colframe=violet!75!black,title=\textbf{Recursive Least Squares}]
New Inverse Covariance:
\begin{equation*}
Q_K \, = \, Q_{k-1} + \phi_K\,\phi_K^T
\end{equation*}
Innovation update:
\begin{equation*}
\hat { \theta  } _{ k }\, =\, \hat { \theta  } _{ k-1 }+\underset { ``innovation'' }{ \underbrace { Q_{ k }^{ -1 }\, \phi _{ k }\, (y_{ k }\, -\, \phi _{ k }^{ T }\, \hat { \theta  } _{ k-1 }) }  } 
\end{equation*}
General Optimization Problem:
\begin{equation*}
\hat { \theta  } _{ k }\, =\, arg\, \underset { \theta  }{ min } \, (\theta -\hat { \theta  } _{ 0 })^{ T }\cdot Q_{ 0 }\cdot (\theta -\hat { \theta  } _{ 0 })+\sum _{ i=1 }^{ k }{ (y_k-\phi_k^T\ \, \cdot \, \theta)^2 } 
\end{equation*}
\end{tcolorbox}

\begin{tcolorbox}[colback=violet!5!white,colframe=violet!75!black,title=\textbf{Kalman Filter}]
Valid for Discrete and Linear!\\
(If recursive least squares: \( x_{ k+1 }\, =\, A_{ k }\cdot x_{ k } \)
\begin{equation*}
x_{ k+1 }\, =\, A_{ k }\cdot x_{ k }+\omega _{ k }\quad and\quad y_{ k }=C_{ k }\cdot x_{ k }+v_{ k }
\end{equation*}
\subsection*{Steps of Kalman Filter}
\begin{description}
\item[1 Prediction] $\hat { x } _{ [k\, |\, k-1] }\, =\, A_{ k-1 }\cdot \hat { x } _{ [k-1\, |\, k-1] }\\ P_{ [k\, |\, k-1] }\, =\, A_{ k-1 }\cdot { P }_{ [k-1\, |\, k-1] }\, \cdot \, A_{ k-1 }^{ T }\cdot { W }_{ k-1 }$ \\
if recursive linear least squares without \( { W }_{ k-1 } \).
\item[2 Innovation update] $P_{ [k\, |\, k] }\, =\, (P_{ [k\, |\, k-1] }^{ -1 }+C_{ k }^{ T }\, \cdot \, { V\,  }^{ -1 }\, \cdot \, C_{ k })^{ -1 }\\ \hat { x } _{ [k\, |\, k] }\, =\, \hat { x } _{ [k\, |\, k-1] }+P_{ [k\, |\, k] }\, \cdot \, C_{ k }^{ T }\, \cdot \, { V\,  }^{ -1 }\, \cdot \, (y_{ k }-C_{ k }\, \cdot \, \hat { x } _{ [k\, |\, k-1] })$
\end{description}
\end{tcolorbox}

