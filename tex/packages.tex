% Packages zum Zeichnen für z. B. Flowchart
\usepackage{tikz}
\usetikzlibrary{shapes,arrows,chains}
\usepackage{verbatim}

% math stuff
\usepackage[nosumlimits, nointlimits, fleqn]{amsmath}
\usepackage{amssymb}

% Zum einfügen von mehrseitigen PDFs
%\usepackage{pdfpages}				

% Zum einfügen von Graphen
\usepackage{pgfplots}
\usepackage{pgfplotstable}

% Für das Abkürzungsverzeichnis notwendig
\usepackage[footnote]{acronym}

% Bildumgebungen
\usepackage{graphicx}
\usepackage{float}
\usepackage{xcolor}	

% Literaturumgebungen
\usepackage{cite}					% Zitieren
\usepackage{bibgerm}				% Literatur in Deutscher DIN

% Tabellenumgebung
\usepackage{longtable}				% Tabellen mit Seitenumbruch
\renewcommand\arraystretch{1.4}   		% Tabellen: Zeilen um Faktor x vergrößern
\usepackage{booktabs}				% professionelle Tabellen (Grundeinstellung bei Excel2LaTeX Excel-PlugIn)

% Sonstige Umgegebungen
%\usepackage{ifthen}		
%\usepackage{caption}
%\usepackage{subcaption}
\usepackage{listings}
%\usepackage{multirow}
%\usepackage{multicol}		
\usepackage[ngerman]{babel}
\usepackage[latin1]{inputenc}
\usepackage[T1]{fontenc}
	

% Blindtext
\usepackage{blindtext}

% Einstellung des Zeilenabstandes z. B. auf der Titelseite
\usepackage{setspace}

% Drei Spalten nebeneinander
\usepackage{multicol}

% Colorboxen
\usepackage{tcolorbox}

% Package um Aufz�hlungen zu "versch�nern"
\usepackage{enumitem}

% Todos hervorheben
\usepackage[obeyDraft]{todonotes}

% Für eine Verlinkung im Inhaltsverzeichnis.
% Dieses Paket möglichst als letztes einbinden falls Fehler auftreten!!!
\usepackage{hyperref}
\hypersetup{
    %bookmarks=true,
    unicode=false,
    pdfborder={0 0 0},
    pdftoolbar=true,
    pdfmenubar=true,
    pdffitwindow=false,
    pdfstartview={FitH},
    pdftitle={HelpSheet},
    pdfnewwindow=true,
    colorlinks=false,
    linkcolor=red,
    citecolor=green,
    filecolor=magenta,
    urlcolor=cyan
}

